\documentclass[runningheads]{llncs}
\usepackage{xr}
\usepackage{color}
\externaldocument{group_management}
\definecolor{codegray}{gray}{0.95}
\newcommand{\code}[1]{\colorbox{codegray}{\texttt{#1}}}

\begin{document}

\subsection{Voting (Leader Election)}

Various tasks within the system are performed by a single server. These tasks include the management of the active group view (see \ref{grpmngmnt}) as well as initiating the byzantine fault tolerance algorithm to ensure system synchronicity. For this reason, an election is required to designate this special instance as the leader of the group.

A leader election can be triggered by three scenarios. The first scenario is the leader deciding to trigger an election when a new server joins the group and the leader deems it necessary. The necessity of an election is decided based on the group view including the new member. Should the leader no longer fulfill the requirement (largest \textit{uid}) an election is deemed necessary. The second scenario is a group member (not leader) not being able to connect to the leader when sending its heartbeat and thus triggering an election. This covers the case that should a leader shut down or crash unexpectedly, the members will pick up on this and trigger an election. The third scenario occurs when a leader shuts down and broadcasts its shutdown message. Any group member receiving this message will trigger a new election. This should be the standard case in which a leader shuts down planned and properly.

The following will describe the algorithm and its implementation used for executing the leader election.

\subsubsection{Ring Implementation} \label{ring_imp}
The group view and its members IDs are known to each member within the group at any given time, due to distribution performed by the leader. The general topology used in this implementation is a \textit{unidirectional clockwise uniform non-anonymous ring}. When required, a member \textit{M1} wanting to start sending or forward an election message to its neighbor \textit{M2}, dynamically builds a local ring based on the group view and sorted in descending order. Sorting is based on the \textit{uid} of each member. \textit{M1} then chooses its neighbor based on its own index \textit{M1i} and adding 1. \textit{M1} then retrieves \textit{M2}s \textit{uid} from the ring structure at the position \textit{M1i + 1}. The ring structure is implemented as a circular list, meaning that if \textit{M1} is the only member of the group, \textit{M1i + 1} will return its own \textit{uid}. As each node builds its own local ring based on the same group using the same sorting method, the ring is built in ascending order by design. This ensures a complexity of \textit{O(n log n)}

\subsubsection{Algorithm Implementation}

The algorithm implemented is the \textit{LeLann-Chang-Roberts} (LCR) algorithm. The leader is defined as the node with the highest \textit{uid}. Each member has the variables \textit{participating} and \textit{current\_leader}. When necessary, a node \textit{M1} will initiate an election by calling its \textit{start\_election} method. This generates the initial \textit{ELECTION\_MESSAGE} containing \textit{M1}s \textit{uid} and \textit{is\_leader} set to \textit{False}. \textit{M1} sets its own \textit{participating} status to \textit{True}. A call to \textit{send\_election\_message} defines the neighbor \textit{M2} as described in above and sends to it the message via TCP.

On arrival of a new election message \textit{E}, a node \textit{M2} will check for one of multiple cases. If \textit{participating is False} and \textit{UID\textsubscript{E}} \textless \textit{my\_uid} then put \textit{my\_uid} onto the message, set \textit{participating = True} and send the message to the next neighbor. Else, if \textit{UID\textsubscript{E}} \textgreater \textit{my\_uid} then set \textit{participating = True} and forward \textit{E} without changing it to my neighbor. Else, if \textit{UID\textsubscript{E}} == \textit{my\_uid} this means that \textit{M2} has received its own election message, meaning that it has the highest \textit{uid}. In this case set \textit{current\_leader = my\_uid}, \textit{participating = False} and \textit{is\_leader\textsubscript{E} = True}. The message is then sent on to the next neighbor.

When a node \textit{M2} receives an election message \textit{E} where \textit{is\_leader\textsubscript{E} == True} it either means that another node has declared itself leader or that the node has received its own leader message. If the node still has \textit{participating = True} it means the leader message has come from another node. \textit{M2} will then set its \textit{current\_leader = UID\textsubscript{E}}, \textit{participating = False} and forward \textit{E} to its next neighbor unchanged. Furthermore, the node will switch its status to \textit{MEMBER} if required. If \textit{M2s} \textit{participating = False} then it means \textit{M2} has received its own leader message. In this case, there is no more work to be done and the election algorithm can be terminated. The newly elected leader switches its mode to \textit{LEADER} and proceeds to distribute the group view and run the byzantine fault tolerance algorithm to ensure synchronicity.

\subsubsection{Failure Management}

Should a node not be able to forward its election message to its next neighbor (as defined by the ring topology), it will remove this neighbor from its own group view and start a new election. The local removal from the group view is necessary to avoid an infinite loop.

To ensure the integrity of the group view, the newly elected leader distributes the group view to all members upon terminating the election algorithm. Before doing so, it cycles through all servers contained in the group view and pings them to ensure they are available. Should a server not be available, it is removed from the group view prior to distribution. This will remove any differences in the group view from members locally removing neighbors in the case they are unreachable.

If a node that has declared itself the leader crashes before it can abort the election on receiving its own leader message, a subsequent node will receive a leader message without itself being marked as \textit{PARTICIPATING}. In this case, the node will abort the current election and start a new one.

A leader, that has for some reason not participated in the election, but is online at the time the new leader pings the current group, will no longer receive heartbeats from the members of the group, meaning that there are two leaders simultaneously. This case is handled in the process described in \ref{rogueleader}.

\end{document}