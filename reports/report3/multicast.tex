\documentclass[runningheads]{llncs}

\begin{document}

\section{Reliable Totally Ordered Multicast} \label{multicast}

Application messages inside the group will be sent via reliable totally ordered
Multicast. In the following, we will describe our implementation and explain
how we handled some corner cases we stumbled across.

\subsection{Procedure} \label{multicastprocedure}

Reliable Totally Ordered Multicast is build upon the IP/Multicast protocol.
Each server has two sockets to realize it, one listener socket which listens on
our defined Multicast address and another socket which is used to send messages
so all messages has a unique sender address. The second socket is needed, so we
know exactly the address who sent a message, so we can answer this server
directly.

If one server sends a message to the group, we add multiple things to the
messages. First we need to give the message a type, so we can distinguish
between messages that are content and messages that are needed for the
protocol. We also add a unique identifier for this message, the unique
identifier of the sender and $S$ of the server. $S$ is needed for reliability
so we know which messages a server missed, so these can be requested for
redelivery.

If the new message was sent, each server listening on the broadcast address,
might now receive this message. First we check if we already received this
message, by comparing the unique message identifier with our received messages.
If we already received the message we can ignore it, if we haven't received it
we put it in our received backlog, and we send a copy of the message to the
group if we aren't the sender of this message. Then we check the piggyback $S$
value and compare it with the current $R$ for that sender. If $S = R + 1$ then
the new message is the next message from this sender, if the $S \leq R + 1$
then the message was already delivered, and we can ignore it and if $S > R + 1$
that means we have missed messages that we either need to request from this
address or messages that could be found in the holdback queue of the server.

We first look at the case where the next message was delivered ($S = R + 1$).
In this case, we then decided to implement ISIS total ordering.
% TODO: Maybe we need to explain here why we need total ordering. Maybe we also
% do this earlier
First we need to set $R$ to $R + 1$, and then we start processing our message
by looking at the type of the message. If the message is a new Application
message we need to propose an order to the original sender by answering with a
$pq = max(aq, pq) + 1$. $aq$ is the largest agreed sequence number and $pq$ is
the largest proposed sequence number. This message will be sent to the second
socket address, not via Multicast, so it only reaches the server which needs
the answers.

This server will then collect these answers from all active members in the
group (we will discuss later how we are handling this with dynamic joining and
leaving servers \ref{multicastchanges}). When all messages are collected the
server will broadcast a new message, with a different type. The content of the
message is the agreed sequence number $a =
max(all\_proposed\_sequence\_numbers)$. We also add a unique ID, sender ID and
the next $S$ to this message, so the message is also reliable.

Upon receiving the message, we compare the $S$ value with the current $R$ the
same way as described above, but we handle the message a different if $S = R +
1$. Now we need to reorder the delivery group based on the received $a$. The
delivery queue is the queue of the next messages that will be delivered, we
have two queues to make it easier for us to handle. A holdback queue of
messages that can't be processed yet because of missing messages and the
delivery queue.

Now we look into what we can do if we notice that we have missed a message. If
$S > R + 1$ we know we have missed at least one message from the sender. We put
this new message in our holdback queue, for later processing and then look
inside the whole back queue if we maybe already received the next message in
the order defined by $S$ and $R$. If that is the case, we process these
messages and remove them from the holdback queue until we find a message for
that $S = R + 1$ is no longer true. We collect all missing $S$ numbers and send
the server a 1 to 1 message that we are missing these $S$ numbers. The server
will then collect these messages and resend them. Both messages aren't
protected with an S because if the message is lost, we run into the next
failure of $S = R + 1$ when the next broadcast message arrives, and we end up
running the algorithm again.

\subsection{Server leaving and joining the group} \label{multicastchanges}

If a new server joins, the group management will make sure that all servers set
up the new member correctly and set $R$ for this server to 0. When the leader
welcomes the new server, it also shares his current $R$ numbers of all the
members of the group and all messages in the delivery queue, so that the new
server can handle a next message that might be an order proposal.

When waiting for proposal messages, the server will not wait for a proposal
from the new joined server, if the server wasn't active when the original
messages was sent. We realize this by keeping a snapshot of the group view at
the time the message was sent. We then only wait for messages of members from
$members_at_time_of_sending \cap current_members$. This also solves the issues
that we no longer wait for messages from server, which already left the group.

\end{document}
