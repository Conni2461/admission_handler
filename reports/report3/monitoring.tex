\documentclass[runningheads]{llncs}

\begin{document}

\subsection{Monitoring and GUIs}

\subsubsection{System Monitor}
Monitoring on a per-node basis is implemented in the form of logging, utilizing log-levels to control the verbosity. A monitoring-server with a simple graphical user interface provides a global overview of the current state of the servers in the system. Various parameters such as server-UUID, currently connected clients, election participation and server state (\textit{LEADED, MEMBER, PENDING}) are displayed. The monitoring server has no direct interaction with the system as it utilizes UDP to listen for specific monitoring data which the individual servers are promoting. To reduce the number of messages being sent on the network, data is only promoted based on events and at regular intervals, namely when a server sends its heartbeat.

\subsubsection{Client GUI}
A basic client interface is implemented for simplified interaction and to demonstrate the intended application of the system. It displays the currently connected server, a count of visitors within the venue and two buttons for interaction: one for a visitor entering the venue and one for a visitor leaving. Upon requesting entry, visual feedback is provided in the case of a confirmation or denial.

\end{document}