\documentclass[runningheads]{llncs}

\begin{document}

\subsection{Byzantine Fault Tolerance} \label{byzantine}

To support fault tolerance, we decided to implement the byzantine algorithm. In
the following, we will describe our implementation and explain how we handled
some corner cases we stumbled across.

\subsubsection{Procedure} \label{byzantineprocedure}

Byzantine algorithm is a synchronous message based algorithm which is used to
implement a form of fault tolerance. To implement this synchronous algorithm in
our asynchronous distributed system we decided to move the system in a state
where we can run this algorithm. For this, we need to pause every application
specific reliable totally ordered multicast~\ref{multicast} message until the
algorithm is completed. This state change is initiated by the leader when a new
server joins. If the system has 4 or more servers, the minimum amount of
servers required for byzantine, the leader sends a pause multicast message via
multicast to all participants in the system. This means that each server pauses
application messages over reliable totally ordered multicast by putting them in
a queue, but not sending them. This allows us to still send internal messages
over reliable totally ordered multicast, for example the resume message. Once
the algorithm is completed, the leader sends a resume multicast message which
means that servers are again permitted to send out multicast messages. So the
queue that hold all application messages, that were not sent out yet will be
emptied.

When all servers are paused, the leader initiates the byzantine algorithm by
sending the byzantine message, named in the following $OM$ to all servers in
the group. The message consists of a $v$ which we chose as the current entries
(application data) on each server, the server destinations, which are all
servers in the group without the leader server, the list of servers which are
passed, on initiation, only the leader, and a $f = \lfloor (n - 1)/ 3 \rfloor$.
This message will be sent to all destinations, as previously defined, via tcp.

On receiving of this message, each server will start a new tree, if not already
started, where it keeps track of all received $OM$ messages. It removes itself
from the list of destinations, adds himself to the beginning of the list of
passed servers, does $f = f - 1$, adds his own entries count as $v$ and packs
everything in a new message $OM'$ which will be sent to all servers left in the
destination list.

This repeats until all messages are sent, and we end up with a tree on each
server that is not the leader. We are then calculating the most common value
for each level, starting at $f - 1$ and for node and select the most common one
out of this list. This most common element will be sent back to the leader
which will then select the most common element from all answers and distribute
the result, with a resume multicast over reliable totally ordered multicast to
all servers. Every server will then take the result as the new value for the
current entries.

If a TCP-Message is not being delivered, we can assume that the connection to
this server is lost and that the group view might have changed. Because of this
we decided to restart the byzantine algorithm if there are still enough server
in the group. This is done by sending a restart message to the current leader
which will then start a new run with a new unique ID. Making the old one
invalid. So if a server receives a byzantine message with a new ID that he
doesn't know he has to discard the current tree and start a new one.

If a new Server wants to join the group while byzantine is running, the leader
answers with a wait for message that signals the new server that there is a
group, but it is not yet permitted to join. The leader will then memorize that
a new server wants to join and will accept that server as soon as the byzantine
algorithm is done.

\end{document}
