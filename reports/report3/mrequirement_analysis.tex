\documentclass[runningheads]{llncs}

\begin{document}

\subsection{Dynamic Discovery} \label{dyndisc}
A fresh server joins the system by broadcasting a join request.
If it does not receive answers, it establishes itself as the new system with zero current guests.
If it does get an answer, the number of arrivals is synchronized between servers.
New clients join the system by broadcasting a request and establishing a TCP connection with the first server to answer them.

\subsection{Ordered Reliable Multicast}
Clients communicate newly arriving guests to their server and only allow passage if they get a positive response.
To ensure a fair entry process and for possible other uses, arrivals are totally ordered.
This is achieved by locking the number of entries upon an entry request using reliable ordered multi-cast.
New arrivals are only permitted if they have been integrated in the total order and are not over capacity according to it.
New entries are also communicated to all connected clients to ensure they display correctly how many free spots are left.

\subsection{Voting}
Is detailed in the implementation part.

\subsection{Fault Tolerance}
If a client does not receive an answer from its associated server, it tries to find a new one via the same process as a freshly joining one.
If a server does not hear from a client, the TCP connection is simply purged.
Each of the servers keeps the order (and number) of entries in memory so it can take over in case the leader fails.
If a server loses an established connection to a system, it notifies connected clients and goes dormant until it is able to reestablish the connection.
This ensures that the venue does not go over capacity from having two independent systems exist.
Further details can be found in their respective implementation sections.

\end{document}