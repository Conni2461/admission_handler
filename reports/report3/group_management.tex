\documentclass[runningheads]{llncs}

\begin{document}

\subsection{Group Management} \label{grpmngmnt}

The current group view is handled and kept up to date by the leader. Group view updates occur in the following cases:
\begin{itemize}
    \item a member joins the group
    \item a member shuts down intentionally and notifies the leader
    \item  a member fails to send a heartbeat to the leader for two intervals in a row.
\end{itemize}
Any changes on the group view trigger the leader to distribute the group view to all members of the group via TCP.

\subsubsection{Heartbeats} \label{heartbeats}

Heartbeats are used to monitor the status of the members of the group view. They are used in a bilateral fashion, meaning that the leader checks for failed members as well as members checking that the leader is still operational.

Heartbeats are sent out by each member of the group to the leader via TCP. This action is performed by a threaded timer. Should the leader have shut down or crashed without notifying the group, the TCP connection will fail and the members of the group will trigger a new election. Parallel to this, the leader performs checks on the heartbeat dictionary it maintains. Also triggered by a threaded timer, the leader will check all registered group members and calculate the time difference between the current time and the latest heartbeat it has received from this member.

Should a group member have failed to send a heartbeat for two intervals in a row, it is marked as failed and removed from the group view. Alternatively, if a heartbeat is received by a server that is not/ no longer in the group view, it will be notified and subsequently trigger a new registration process. If a heartbeat is received by a server that is not the leader, an election is triggered.

\end{document}